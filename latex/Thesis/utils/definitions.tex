% File to store all kind of Defintions ...
%%%%%%%%%%%%%%%%%%%
%% definitions
%%%%%%%%%%%%%%%%%%%
\graphicspath{{fig}}

\def\BaAuthor{Max Mustermann}
\def\BaTitle{Titel Titel Titel Titel}
\def\BaSupervisorOne{Prof.\ Dr.\ Albert Einstein}
\def\BaSupervisorTwo{Prof.\ Dr.\ Konrad Zuse}
\def\BaDeadline{17.12.2017}



\newcommand{\chapterref}[1]{Kapitel~\ref{#1}}
\newcommand{\sectionref}[1]{Teilabschnitt~\ref{#1}}
\newcommand{\imageref}[1]{Abbildung~\ref{#1}}
\newcommand{\pageWithNameRef}[1]{\emph{\nameref{#1}}}
\newcommand{\chapterWithNameRef}[1]{\chapterref{#1} --~\pageWithNameRef{#1}}
\newcommand{\sectionWithNameRef}[1]{\sectionref{#1} --~\pageWithNameRef{#1}}
\newcommand{\tableref}[1]{Tabelle~\ref{#1}}
\newcommand{\tableWithNameRef}[1]{\tableref{#1} auf~\pageWithNameRef{#1}}
\newcommand{\ideascrum}{Idea\-Scrum~}
\newcommand{\imageWithNameRef}[1]{\imageref{#1} --~\pageWithNameRef{#1}}
\newcommand{\sectionStarWithNameRef}[1]{Teilabschnitt~\emph{\nameref{#1}}}





%\hypersetup{
%pdfauthor={\BaAuthor},
%pdftitle={\BaTitle},
%pdfsubject={Choose your Subject wisely ... },
%pdfkeywords={Lorem, Ipsum, etc...}
%}
%----------------------------------------------------------------------------------------
%	HYPERLINKS IN THE DOCUMENTS
%----------------------------------------------------------------------------------------

\hypersetup{hidelinks,backref=true,pagebackref=true,hyperindex=true,colorlinks=false,breaklinks=true,urlcolor= timoorange,bookmarks=true,bookmarksopen=false,pdftitle={Title},pdfauthor={Author}}
\usepackage{bookmark}
\bookmarksetup{
open,
numbered,
addtohook={%
\ifnum\bookmarkget{level}=0 % chapter
\bookmarksetup{bold}%
\fi
\ifnum\bookmarkget{level}=-1 % part
\bookmarksetup{color=timoorange,bold}%
\fi
}
}



%----------------------------------------------------------------------------------------
%	PART HEADINGS
%----------------------------------------------------------------------------------------
% Part text styling
\titlecontents{part}[0cm]
{\addvspace{20pt}\centering\large\bfseries}
{}
{}
{}


%% numbered part in the table of contents
%\newcommand{\@mypartnumtocformat}[2]{%
%\setlength\fboxsep{0pt}%
%\noindent\colorbox{timoorange!20}{\strut\parbox[c][.7cm]{\ecart}{\color{timoorange!70}\Large\sffamily\bfseries\centering#1}}\hskip\esp\colorbox{timoorange!40}{\strut\parbox[c][.7cm]{\linewidth-\ecart-\esp}{\Large\sffamily\centering#2}}}%
%%%%%%%%%%%%%%%%%%%%%%%%%%%%%%%%%%%
%% unnumbered part in the table of contents
%\newcommand{\@myparttocformat}[1]{%
%\setlength\fboxsep{0pt}%
%\noindent\colorbox{timoorange!40}{\strut\parbox[c][.7cm]{\linewidth}{\Large\sffamily\centering#1}}}%
%%%%%%%%%%%%%%%%%%%%%%%%%%%%%%%%%%
\newlength\esp
\setlength\esp{4pt}
\newlength\ecart
\setlength\ecart{1.2cm-\esp}
\newcommand{\thepartimage}{}%
\newcommand{\partimage}[1]{\renewcommand{\thepartimage}{#1}}%
\def\@part[#1]#2{%
\ifnum \c@secnumdepth >-2\relax%
\refstepcounter{part}%
\addcontentsline{toc}{part}{\texorpdfstring{\protect\@mypartnumtocformat{\thepart}{#1}}{\partname~\thepart\ ---\ #1}}
\else%
\addcontentsline{toc}{part}{\texorpdfstring{\protect\@myparttocformat{#1}}{#1}}%
\fi%
\startcontents%
\markboth{}{}%
{\thispagestyle{empty}%
\begin{tikzpicture}[remember picture,overlay]%
\node at (current page.north west){\begin{tikzpicture}[remember picture,overlay]%	
\fill[timoorange!20](0cm,0cm) rectangle (\paperwidth,-\paperheight);
\node[anchor=north] at (4cm,-3.25cm){\color{timoorange!40}\fontsize{220}{100}\sffamily\bfseries\@Roman\c@part}; 
\node[anchor=south east] at (\paperwidth-1cm,-\paperheight+1cm){\parbox[t][][t]{8.5cm}{
\printcontents{l}{0}{\setcounter{tocdepth}{1}}%
}};
\node[anchor=north east] at (\paperwidth-1.5cm,-3.25cm){\parbox[t][][t]{15cm}{\strut\raggedleft\color{white}\fontsize{30}{30}\sffamily\bfseries#2}};
\end{tikzpicture}};
\end{tikzpicture}}%
\@endpart}
\def\@spart#1{%
\startcontents%
\phantomsection
{\thispagestyle{empty}%
\begin{tikzpicture}[remember picture,overlay]%
\node at (current page.north west){\begin{tikzpicture}[remember picture,overlay]%	
\fill[timoorange!20](0cm,0cm) rectangle (\paperwidth,-\paperheight);
\node[anchor=north east] at (\paperwidth-1.5cm,-3.25cm){\parbox[t][][t]{15cm}{\strut\raggedleft\color{white}\fontsize{30}{30}\sffamily\bfseries#1}};
\end{tikzpicture}};
\end{tikzpicture}}
\addcontentsline{toc}{part}{\texorpdfstring{%
\setlength\fboxsep{0pt}%
\noindent\protect\colorbox{timoorange!40}{\strut\protect\parbox[c][.7cm]{\linewidth}{\Large\sffamily\protect\centering #1\quad\mbox{}}}}{#1}}%
\@endpart}
\def\@endpart{\vfil\newpage
\if@twoside
\if@openright
\null
\thispagestyle{empty}%
\newpage
\fi
\fi
\if@tempswa
\twocolumn
\fi}
